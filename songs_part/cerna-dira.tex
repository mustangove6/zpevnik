\beginsong{Černá díra}[by=Karel Plíhal]
\beginverse
\[D]Mívali jsme \[A]dědečka, \[G]starého už \[A]pá\[D]na,
\[D]stalo se to \[A]v červenci \[G]jednou časně \[A]zrá\[D]na,
\[Hmi]šel do sklepa \[G]pro vidle, \[E]aby seno \[A]sklízel,
\[D]už se ale \[A]nevrátil, \[G]prostě někam \[A]zmi\[D]zel.
\endverse
\beginverse
Máme doma ve sklepě malou černou díru,
na co přijde, sežere, v ničem nezná míru,
nechoď, babi, pro uhlí, sežere i tebe,
už tě nikdy nenajdou příslušníci VB.
\endverse
\beginverse
Přišli vědci zdaleka, přišli vědci zblízka,
babička je nervózní a nás, děti, tříská,
sama musí poklízet, běhat kolem plotny,
a děda je ve sklepě nekonečně hmotný.
\endverse
\beginverse
Hele, babi, nezoufej, moje žena vaří
a jídlo se jí většinou nikdy nepodaří,
půjdu díru nakrmit zbytky od oběda,
díra všechno vyvrhne, i našeho děda.
\endverse
\beginverse
Tak jsem díru nakrmil zbytky od oběda,
díra všechno vyvrhla, i našeho děda,
potom jsem ji rozkrájel motorovou pilou,
opět člověk zvítězil nad neznámou silou.
\endverse
\beginverse
\[E]Dědeček se \[H]raduje, \[A]že je zase \[E]v penzi,
\[E]teď je naše \[H]písnička \[A]zralá pro re\[H]cen\[E]zi.
\endverse
\endsong