\beginsong{Sáro}[by=Traband]
\beginchorus
\[Ami]Sáro, \[Emi]Sáro, \[F]v noci se mi \[C]zdálo,
že \[F]tři andělé \[C]Boží k nám \[F]přišli na o\[G]běd.
\[Ami]Sáro, \[Emi]Sáro, \[F]jak moc a nebo \[C]málo,
mi \[F]chybí abych \[C]Tvojí duši \[F]mohl rozu\[G]mět?
\endchorus
\beginverse
\[Ami]Sbor kajícných \[Emi]mnichů jde \[F]krajinou v \[C]tichu,
a pro \[F]všechnu lidskou \[C]pýchu má jen \[F]přezíravý \[G]smích.
\[Ami]A z prohraných \[Emi]válek se \[F]vojska domů \[C]vrací,
ač \[F]zbraně stále \[C]burácí, a \[F]bitva zuří v \[G]nich.
\endverse
\beginchorus  \endchorus
\beginverse
Vévoda v zámku čeká na balkóně,
až přivedou mu koně, pak mává na pozdrav.
A srdcová dáma má v každé ruce růže,
Tak snadno pohřbít může, sto urozených hlav.
\endverse
\beginchorus  \endchorus
\beginverse
Královnin šašek s pusou od povidel,
sbírá zbytky jídel, a myslí na útěk.
A v podzemí skrytí slepí alchymisté,
už objevili jistě proti povinnosti lék.
\endverse
\beginchorus
Sáro, Sáro, v noci se mi zdálo,
že tři andělé k nám přišli na oběd.
Sáro, Sáro, jak moc a nebo málo,
Ti chybí abys mojí duši mohla rozumět?
\endchorus
\beginverse
Páv pod tvým oknem zpívá sotva procit,
o tajemstvích noci ve tvých zahradách.
A já - potulný kejklíř, co svázali mu ruce,
teď hraju o tvé srdce a chci mít Tě nadosah.
\endverse
\beginchorus
Sáro, Sáro, pomalu a líně,
s hlavou na Tvém klíně chci se probouzet.
Sáro, Sáro, Sáro, Sáro, rosa padá ráno,
a v poledne už možná bude jiný svět.
Sáro, Sáro, vstávej, milá Sáro!
Andělé k nám přišli na oběd.
\endchorus
\endsong
