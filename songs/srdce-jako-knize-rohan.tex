\beginsong{Srdce jako kníže Rohan}[by=Richard Müller]
\beginverse
\[F]Měsíc je jak Zlatá bula \[C]Sicílska
\[Ami]Stvrzuje se že kto chce ten se \[G]dopíská
\[F]pod lampou jen krátce v přítmí \[C]dlouze zas
\[Ami]Otevře ti Kobera a \[G]můžeš mezi \[F]nás 
\[C Ami G F C Ami G]...
\endverse
\beginverse
Moje teta, tvoje teta, parole
dvaatřicet karet křepčí na stole
měsíc svítí sám a chleba nežere
Ty to ale koukej trefit frajere.
Protože
\endverse
\beginchorus
\[F]Dnes je valcha u starýho \[C]Růžičky
\[Ami]dej si prachy do pořádny \[G]ruličky
\[F]Co je na tom že to není \[C]extra nóbl byt
\[Ami]Srdce jako kníže Rohan \[G]musíš mít \[F]\[C]\[Ami]\[G]
\endchorus
\beginverse
Ať si přes den docent nebo tunelář
herold svatý pravdy nebo inej lhář
tady na to každej kašle zvysoka
pravda je jen jedna - slova proroka říkaj že:
\endverse
\beginchorus
Když je valcha u starýho Růžičky
budou v celku nanič všechny řečičky
Buď to trefa nebo kufr - smůla nebo šnit
jen to srdce jako Rohan musíš mít
\endchorus
\beginverse
Kdo se bojí má jen hnědý kaliko
možná občas nebudeš mít na mlíko
jistě ale poznáš co si vlastně zač
svět nepatřil nikomu kdo nebyl hráč
\endverse
\beginchorus
Ať je valcha u starýho Růžičky
nebo pouť až k tváři Boží rodičky
Ať je valcha, červen, mlha, bouřka nebo klid
Srdce jako kníže Rohan musíš mít
\endchorus
\beginchorus
Dnes je valcha u starýho Růžičky
když si malej tak si stoupni na špičky
malej nebo nachlapenej Cikán, Brňák, Žid
Srdce jako kníže Rohan musíš mít
\endchorus
\beginchorus
Dnes je valcha u starýho Růžičky (to víš že jo ..)
dej si prachy do pořádny ruličky
Co je na tom že to není extra nóbl byt
Srdce jako kníže Rohan musíš mít
\endchorus
\beginchorus
Dnes je valcha u starýho Růžičky
dej si prachy do pořádny ruličky
Co je na tom že to není extra nóbl byt
Srdce jako kníže Rohan musíš mít
\endchorus
\beginchorus
Ať je valcha u starýho Růžičky
nebo to uďáš k tváři Boží rodičky
Ať je valcha, červen, mlha, bouřka nebo klid
Srdce jako kníže Rohan musíš mít 
\endchorus
\endsong
