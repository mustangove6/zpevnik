Ve vědeckých aplikacích je někdy nutné ohřát vzorek zavěšený ve vakuu. Využívá se k tomu proudu urychlených elektronů, jejichž pohybová energie se při dopadu mění na tepelnou. Mějme katodu s emisním proudem elektronů $I\_e$, které jsou urychlovány napětím $U$. Takto urychlené elektrony dopadají na vzorek o povrchu $S$,(předpokládejme, že je dost malý, aby byla teplota všude stejná, a že veškerá energie elektronů se přemění na teplo). Na jaké teplotě se vzorek ustálí, předpokládáme-li, že se chová jako dokonale černé těleso a teplo ztrácí pouze vyzařováním?                                 Heslo: kacenkajeprincezna
Předpokládáme konstantní emisní proud, tedy $I\_e=Q/t$. Energie náboje urychleného napětím $U$ je $E=QU=UI\_et$, výkon ohřívající vzorek je tedy $P=UI\_e$. Pro výkon vyzařovaný do okolí platí $P\_{ok}=\sigma ST^4$. Vzorek se ustálí na takové teplotě, kdy $P=P_ok$, tedy $UI\_e=\sigma ST^4$, odkud přímo $T=\sqrt[4]{\frac{UI\_e}{\sigma S}}$. 