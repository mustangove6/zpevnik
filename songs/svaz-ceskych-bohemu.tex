\beginsong{Svaz českých bohémů}[by=Wohnout]
\beginchorus
\[G]Vracím se domů nad rá\[Dmi]nem, kvalitním vínem omá\[Ami]men, 
z přímek se stávaj zatá\[F]čky, točí se \[C]svět, \brk jsem na sra\[G]čky. 
Vedle mě zvrací prince\[Dmi]zna, nastavím \brk dlaň a pože\[Ami]hnám, 
pro všechny jasný posel\[F]ství - tomu se \[C]říká přátel\[G]ství. 
\endchorus 
\beginverse
\[G]Mám tisíc otá\[Dmi]zek na žádnou vzpomín\[Ami]ku,       
skládám si obrá\[F]zek kám\[C]en po kamí\[G]nku.        
Včerejší produk\[Dmi]ce šla do bezvědo\[Ami]mí, 
nastává deduk\[F]ce, co \[C]na to svědo\[G]mí. 
\endverse
\beginverse
^A už si vzpomí^nám, já byl jsem na sra^zu, 
s kumpány který ^mám, pat^říme do sva^zu, 
vlastníme domé^nu, tak si nás rozklik^ni, 
svaz českejch bohé^mů... 
\endverse
\beginchorus \endchorus
\beginverse
^Stačí pár večír^ků, společně s bohé^my, 
za kterými se ^táhnou od ^školy problé^my. 
V partě je Bleko^ta, Jekota, Meko^ta, 
dost často hovo^říme o ^smyslu živo^ta.
\endverse
\beginverse
^Jako je třeba ^teď, mám tisíc otá^zek, 
na žádnou vzpomín^ku, si ^skládám obrá^zek. 
Z těžkejch ran lížu ^se, včera jsme slavi^li, 
svatýho Vyšu^se. 
\endverse
\beginchorus \endchorus
\beginverse*
^Svět zrychluje svý otáč^ky, sousedka \brk peče koláč^ky. 
Hlášen stav nouze nejvyš^ší, Hapkové \brk ^volaj Horá^čky. 
Zástupy českejch bohé^mů, vyráží ven do teré^nů 
šavlí z kvalitního ví^na, bojovat ^proti systé^mu. 
\endverse
\beginverse*
^Tak jsme se tu všichni sešli, ^co myslíš, osobo? 
^Na to nelze říci než "^Co je ti ^do toho?" 
^Tak jsme se tu všichni sešli, ^co myslíš, osude? 
^Na to nelze říci než "^Jinak to ^nebude "
\rep{3}
\endverse

\endsong