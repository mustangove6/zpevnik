\beginsong{Svaz českých bohémů}[by=Wohnout]
\beginchorus
\[G]Vracím se domů nad rá\[Dmi]nem, kvalitním vínem omá\[Ami]men, 
z přímek se stávaj zatá\[F]čky, točí se \[C]svět, jsem na sra\[G]čky. 
Vedle mě zvrací prince\[Dmi]zna, nastavím dlaň a pože\[Ami]hnám, 
pro všechny jasný posel\[F]ství - tomu se \[C]říká přátel\[G]ství. 
\endchorus 
\beginverse
Mám tisíc otázek na žádnou vzpomínku,       
skládám si obrázek kámen po kamínku.        
Včerejší produkce šla do bezvědomí, 
nastává dedukce, co na to svědomí. 
\endverse
\beginverse
A už si vzpomínám, já byl jsem na srazu, 
s kumpány který mám, patříme do svazu, 
vlastníme doménu, tak si nás rozklikni, 
svaz českejch bohémů... 
\endverse
\beginchorus \endchorus
\beginverse
Stačí pár večírků, společně s bohémy, 
za kterými se táhnou od školy problémy. 
V partě je Blekota, Jekota, Mekota, 
dost často hovoříme o smyslu života.
\endverse
\beginverse
Jako je třeba teď, mám tisíc otázek, 
na žádnou vzpomínku, si skládám obrázek. 
Z těžkejch ran lížu se, včera jsme slavili, 
svatýho Vyšuse. 
\endverse
\beginchorus \endchorus
\beginverse*
Svět zrychluje svý otáčky, sousedka peče koláčky. 
Hlášen stav nouze nejvyšší, Hapkové volaj Horáčky. 
Zástupy českejch bohému, vyráží ven do terénů 
šavlí z kvalitního vína, bojovat proti systému. 
\endverse
\beginverse*
Tak jsme se tu všichni sešli, co myslíš, osobo? 
Na to nelze říci než "Co je ti do toho?" 
Tak jsme se tu všichni sešli, co myslíš, osude? 
Na to nelze říci než "Jinak to nebude." \rep{3}
\endverse

\endsong